%/*************************************************************************
% *
% * FILE  basicusage.tex
% * $Id: basicusage.tex,v 1.12 2001/09/22 20:38:15 tlr Exp $
% *
% * DESCRIPTION 
% *   The basic usage chapter.
% *
% * This file is part of ztracker - a tracker-style MIDI sequencer.
% *
% * Copyright (c) 2001, Daniel Kahlin <tlr@users.sourceforge.net>
% * All rights reserved.
% *
% * Redistribution and use in source and binary forms, with or without
% * modification, are permitted provided that the following conditions
% * are met:
% * 1. Redistributions of source code must retain the above copyright
% *    notice, this list of conditions and the following disclaimer.
% * 2. Redistributions in binary form must reproduce the above copyright
% *    notice, this list of conditions and the following disclaimer in the
% *    documentation and/or other materials provided with the distribution.
% * 3. Neither the names of the copyright holders nor the names of their
% *    contributors may be used to endorse or promote products derived 
% *    from this software without specific prior written permission.
% *
% * THIS SOFTWARE IS PROVIDED BY THE COPYRIGHT HOLDERS AND CONTRIBUTORS
% * ``AS IS�� AND ANY EXPRESS OR IMPLIED WARRANTIES, INCLUDING, BUT NOT
% * LIMITED TO, THE IMPLIED WARRANTIES OF MERCHANTABILITY AND FITNESS FOR
% * A PARTICULAR PURPOSE ARE DISCLAIMED.  IN NO EVENT SHALL THE REGENTS OR
% * CONTRIBUTORS BE LIABLE FOR ANY DIRECT, INDIRECT, INCIDENTAL, SPECIAL,
% * EXEMPLARY, OR CONSEQUENTIAL DAMAGES (INCLUDING, BUT NOT LIMITED TO,
% * PROCUREMENT OF SUBSTITUTE GOODS OR SERVICES; LOSS OF USE, DATA, OR
% * PROFITS; OR BUSINESS INTERRUPTION) HOWEVER CAUSED AND ON ANY THEORY OF
% * LIABILITY, WHETHER IN CONTRACT, STRICT LIABILITY, OR TORT (INCLUDING
% * NEGLIGENCE OR OTHERWISE) ARISING IN ANY WAY OUT OF THE USE OF THIS
% * SOFTWARE, EVEN IF ADVISED OF THE POSSIBILITY OF SUCH DAMAGE.
% *
% ******/
\chapter{Basic Usage}
%\begin{figure}
%\includegraphics[width=10cm]{pics/patterneditor.png}
%\end{figure}

\section{The Pattern Editor}
Pressing \key{F2} bring up the pattern editor.
Let's begin by a showing a row in the pattern editor set to \emph{View mode:~Big}\ldots{}
\begin{verbatim}
E-2 00 7F 024 W2000
\end{verbatim}
\begin{itemize}
\item [\texttt{E-2}] is the note itself.  In this case an E played in octave 2.
\item [\texttt{00}] is the instrument number.
\item [\texttt{7F}] is the velocity of this note.
\item [\texttt{024}] is the length of the note in subticks.
\item [\pc{W2000}] is a \emph{player command}. This particular command sets the
pitch bend to the center position.
\end{itemize}
The same row looks like this in \emph{View mode:~Regular}\ldots{}
\begin{verbatim}
E-2 00 7F 024
\end{verbatim}
\ldots{}like this in \emph{View mode:~FX}\ldots{}
\begin{verbatim}
E-2 7F W2000
\end{verbatim}
\ldots{}and like this in \emph{View mode:~Squish}
\begin{verbatim}
E-2 7F
\end{verbatim}

\section{Player Commands}

\subsubsection{Command \pc{A..xx}}
The player command \pc{A..xx} will set the ticks per beat value to \pc{xx},
where \pc{xx} is the tpb value in hexadecimal numbers.
The only allowed tpb values are 2 (\pc{02}), 4 (\pc{04}), 6 (\pc{06}), 8 (\pc{08}),
12 (\pc{0C}), 24 (\pc{18}) and 48 (\pc{30}).

\subsubsection{Command \pc{C..xx}}
When the player command \pc{C..xx} is encountered, the next tick the player
will skip to the next pattern in queue from row \pc{xx}, where \pc{xx} is the
row in hexadecimal numbers.

\subsubsection{Command \pc{D..xx}}
The player command \pc{D..xx} delays the current note, volume or note cut with
\pc{xx} subticks, where \pc{xx} is the number of subticks in hexadecimal
numbers.

\subsubsection{Command \pc{Exxxx}}
The player command \pc{Exxxx} makes a pitch slide down.  The pitch bend value
is decremented by \pc{xxxx} every subtick.  \pc{E0000} repeats the parameter
from the pitch slide on the previous row.

\subsubsection{Command \pc{Fxxxx}}
The player command \pc{Fxxxx} makes a pitch slide up.  The pitch bend value is
incremented by \pc{xxxx} every subtick.  \pc{F0000} repeats the parameter from
the pitch slide on the previous row.

\subsubsection{Command \pc{P....}}
The player command \pc{P....} will send a MIDI program change message corresponding
to the settings in the current instrument.

\subsubsection{Command \pc{Q..xx}}
The player command \pc{Q..xx} will retrigger the note once every \pc{xx} subticks.

\subsubsection{Command \pc{R..xx}}
The player command \pc{R..xx} will start the arpeggio \pc{xx}.  If \pc{xx} is \pc{00}
the last started arpeggio will be continued.

\subsubsection{Command \pc{Sxxyy}}
The player command \pc{Sxxyy} sets the MIDI continuous controller \pc{xx} to
the value \pc{yy}, where \pc{xx} and \pc{yy} are hexadecimal numbers.  If
\pc{xx} or \pc{yy} are \pc{80} their respective last used values will be used
again.

\subsubsection{Command \pc{T..xx}}
The player command \pc{T..xx} will set the tempo to \pc{xx}, where \pc{xx} is
the tempo in hexadecimal numbers.  The minimum allowed tempo is 60 bpm
(\pc{3C}), and the maximum tempo is 240 bpm (\pc{F0}).

\subsubsection{Command \pc{Wxxxx}}
The player command \pc{Wxxxx} sets the pitch bend to \pc{xxxx}, where \pc{xxxx}
is the desired value in hexadecimal numbers.  \pc{W0000} is maximum down,
\pc{W3FFF} is maximum up, and \pc{W2000} is the center position.  \pc{W2000}
is typically used to reset the pitch bend after a pitch slide command.

\subsubsection{Command \pc{X..xx}}
The player command \pc{X..xx} will set panning to \pc{xx}, where \pc{xx} is
the panning in hexadecimal numbers.  \pc{00} is far left, \pc{7F} is far
right, and \pc{40} is center. (this command does the same as \pc{S0Axx})

\subsubsection{Command \pc{Zxxyy}}
The player command \pc{Zxxyy} will send the midimacro \pc{xx}, with the optional
parameter \pc{yy}.  If \pc{xx} is \pc{00} the last used midimacro will be used again.
If \pc{yy} is \pc{00} the last used value will be used again.

\begin{table}
\newlength{\pcwidth}
\settowidth{\pcwidth}{\pc{Xxxxx}}
\begin{tablist}{\pcwidth}
\hline
\pc{A..xx} & Set TPB (ticks per beat) \\

\pc{Cxxxx} & Pattern break - setting the \pc{C} marker denotes the end of a 
         pattern.  ex: \pc{C0000} will skip to the next pattern, row 0 \\
 
\pc{D..xx} & Delay note/volume/cut - by \pc{xxxx} sub-ticks. \\
 
\pc{Exxxx} & Pitch slide down, \pc{E0000} will repeat the last portamento command. \\
 
\pc{Fxxxx} & Pitch slide up, \pc{F0000} will repeat the last portamento down command. \\
 
\pc{P....} & Send program change message for the current instrument. \\
 
\pc{Q..xx} & Retrig note every \pc{xx} subticks \\
 
\pc{R..xx} & Start arpeggio \pc{xx}.  \pc{R0000} will continue the last arpeggio \\

\pc{Sxxyy} & \pc{xx} is the CC (Continuous Controller) number, \pc{yy} is the value
             (\pc{00}-\pc{80}).  ex. \pc{S0142} will set CC \pc{01} (Mod Wheel) to a
             value of \pc{42}.  $>=$\pc{80} sends last value \\
 
\pc{T..xx} & Set tempo to \pc{xx} bpm \\
 
\pc{Wxxxx} & Absolute pitch set, between \pc{0000} and \pc{3FFF}, \pc{2000} is no pitch change, 
         \pc{0000} is max down, \pc{3FFF} is max up  \\
 
\pc{X..xx} & Set panning, \pc{00}=left, \pc{40}=center, \pc{7F}=right \\

\pc{Zxxyy} & Send midimacro \pc{xx} with parameter \pc{yy}.
             \pc{00} repeats the last used value \\

\hline
\end{tablist}
\caption[Player Commands]{Player Commands}
\end{table}

\section{Keyboard Shortcuts}
These are the key commands.  Note that name of the keys is from an american keyboard layout,
but it is their position that counts.  I.e some key names are different if you have a
keyboard layout of another country.

\newlength{\keywidth}
\setlength{\keywidth}{3cm}

\subsection{Pattern Editing Keys}
\begin{tablist}{\keywidth}
\hline
\key{Space}      & Turns Edit (or "Keyjazz") mode on or off \\
\key{Home/End}   & Just try them \\
\key{PgUp/PgDn}  & Up/down 16 rows \\
\key{Ins/Del}    & Insert/delete row on current track 
                   Hold \key{CTRL} for ALL tracks \\
 
\key{Tab}        & Next track \\
\key{SHIFT-Tab}  & Previous track \\ 
\key{CTRL-Tab}   & Change pattern edit viewmode \\
 
\key{ALT-1 $=>$ 0} & Mute/unmute track 1-10 \\
\key{ALT-F9}     & Mute current track \\
\key{ALT-F10}    & Solo current track \\
 
\key{keypad +}   & Moves ahead one pattern. \\
\key{keypad -}   & Moves back one pattern. \\
\key{SHIFT-keypad +} & Moves ahead one order \\
\key{SHIFT-keypad -} & Moves back one order \\
\key{keypad /}   & Octave down. \\
\key{keypad *}   & Octave up. \\
 
\key{SHIFT-,}    & Select previous instrument. \\
\key{SHIFT-.}    & Select next instrument. \\
\key{SHIFT-~}    & Switch between regular and effect-draw mode 
                   Use \key{Tab} here to switch edit mode \\
 
\key{CTRL-1 $=>$ 0} & Change the step-value in the pattern editor \\
\key{ALT-`}       & Set previous note's length to distance between 
                    note and current cursor position \\
\key{ScrollLock}  & Pattern-follow mode (cursor follows
                    playback) \\
\hline
\end{tablist}

\subsection{Block Functions Keys}
\begin{tablist}{\keywidth}
\hline
\key{CTRL-B}     & Marks the beginning of a select block. \\
\key{CTRL-E}     & Marks the end of a select block. \\
\key{CTRL-L}     & Select whole track, pressing it again will select the entire 
                   pattern. \\
\key{CTRL-U}     & Unselect block \\
\key{CTRL-C}     & Copy the selected block to the clipboard. \\
\key{CTRL-P}     & Copy pattern, paste to next empty pattern, 
                   and jump to new pattern  \\
\key{CTRL-V}     & Paste the contents of the clipboard to the cursor location. \\
\key{CTRL-O}     & Overwrite paste \\
\key{CTRL-M}     & Merge paste the contents of the clipboard with the contents 
                   of the location you're pasting to. \\
\key{CTRL-X}     & Cut the selected block. (remove and copy to clipboard) \\
\key{CTRL-Z}     & Clear block (hold this one) \\
 
\key{CTRL-N}     & Set length of first row of block to length of block \\
\key{CTRL-W}     & Clear unused volumes \\
\key{CTRL-I}     & Interpolate effect data thru block \\
\key{CTRL-T}     & Set all effect and effect data fields of block to same value
                   as first row in block\\
\key{CTRL-K}     & Interpolate volume (block) \\
\key{ALT-Q}      & Transpose block up \\
\key{ALT-A}      & Transpose block down \\
\key{CTRL-J}     & Popup scale volume window (10-200 \%) \\
\key{CTRL-`}     & Set all note's lengths in selection to distance 
                   between note and next note \\
\key{ALT-S}      & Set instruments in block to current instrument \\
\key{SHIFT-move} & Block selection (movements are arrows and \key{PgUp}/\key{PgDn}) \\
\hline
\end{tablist}

\subsection{Instrument Editor Keys}
\begin{tablist}{\keywidth}
\hline
\key{ALT-1 $=>$ 1}   & Quick selects midi device \\
\key{CTRL-1 $=>$ 9}  & Quick selects midi channel \\
\key{ALT-T}          & Toggle tracker mode on/off \\
\key{Mouse-2}        & Focus a slider \\
\hline
\end{tablist}

\subsection{Order Editor Keys}
\begin{tablist}{\keywidth}
\hline
\key{SHIFT-keypad +} & Moves current order ahead \\
\key{SHIFT-keypad -} & Moves current order back \\
\hline
\end{tablist}

\subsection{Global Keys}
\begin{tablist}{\keywidth}
\hline
\key{F1}         & Quick Help \\
\key{F2}         & Pattern editor - Pressing \key{F2} again while in the pattern 
                   editing screen brings up the pattern settings. \\
\key{F3}         & Instrument editor \\
\key{F4}         & Arpeggio editor \\
\key{CTRL-F4}    & Midimacro editor \\
\key{F10}        & Song message editor \\
\key{F5}         & Song play (\key{F5} PatternDisplay widget, 
                   \key{Left}-\key{Right}/\key{Space}/\key{ALT-1} thru \key{ALT-0}/\key{ALT-F9}/\key{ALT-F10})  \\
\key{F6}         & Pattern Play - will play the current pattern being edited. \\
\key{F7}         & Start the song at the current row -  
              If in the pattern edit mode, \key{F7} will start at the current 
              row, at the first instance of the pattern in the pattern 
              order.  If in the pattern order screen, \key{F7} will start the 
              song at the cursor position (by pattern).  \\
\key{SHIFT-F11}  & Set current order \\
\key{SHIFT-F7}   & Play song from the current order  \\
\key{F8}         & Stop playback  \\
\key{F9}         & Panic (all midi-off/\key{SHIFT-F9} performs a hard driver panic) \\
\key{CTRL-F9}    & Load Song... \\
\key{CTRL-F10}   & Save Song As... \\
\key{F11}        & Song Configuration and Order editor \\
\key{F12}        & System Configuration (midi dev select)  \\
\key{ALT-F12}    & About  \\
\key{CTRL-S}     & Save Song \\
\key{CTRL-ALT-N} & New Song \\
\key{ALT-P}      & Song Duration \\
\key{CTRL-ALT-Q} & Quit \\
\hline
\end{tablist}

\subsection{User Interface Keys}
\begin{tablist}{\keywidth}
\hline
\key{Up/Down}    & Cycles through widgets \\
\key{Tab}        & Cycles forwards through widgets \\
\key{SHIFT-Tab}  & Cycles backwards through widgets \\
\key{Enter}      & Confirms choice \\
\key{Space}      & Toggles an option or clicks a button \\
 
\key{Sliders}    & If you hold \key{CTRL} while using \key{Left}/\key{Right} the slider
                   will move in bigger steps \\
\key{Listboxes}  & \key{Up}/\key{Down} scroll, \key{Space} toggles select-items in a list box \\
\key{Popups}     & \key{Esc} closes, all other UI keys apply \\
\hline
\end{tablist}


\section{Pages}
\notyet{}

\section{Editing Song}
\notyet{}

\section{Editing Tracks}
\notyet{}

\section{Loading}
Pressing \key{CTRL-F9} switches to the load page.  Here you may select a file for
loading.  Press \key{Enter}, or click twice on the file you wish to load.  If you
have edited the song currently in memory, you will be asked if you wish to lose
those edits.  \ztracker{} loads \fname{.zt} files.  Further if the file you select
has the \fname{.it} extension, \ztracker{} will try to import it as an
Impulse Tracker song.  \sectionref{sec:importing_it_songs}

\section{Saving}
Pressing \key{CTRL-F10} switches to the save page.  Here you may select a
directory and file name for your song file.  Find the correct directory by using the
cursor keys and \key{Tab} to cycle through the windows (or use the mouse).  Then
enter the desired file name into the text box below the file name selector, and
press \key{Enter} to save. If the file already exists, you will be asked if you
wish to overwrite the previous file.    You can also select an already existing
file by selecting it as in the load page. \\
There are two boxes just below the file name text box, selecting if a 
\fname{.zt} file or a \fname{.mid} file shall be created.  This must always be
\fname{.zt} for \ztracker{} to be able to load the resulting file. Selecting
\fname{.mid} makes \ztracker{} export a MIDI file.
\sectionref{sec:exporting_mid_files} 

% eof
